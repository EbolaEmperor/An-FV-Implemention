\documentclass[lang=cn,10pt]{elegantbook}

\title{Project: 四阶有限体积方法}

\author{Wenchong Huang}
\date{Aug 20, 2023}

\setcounter{tocdepth}{3}

\logo{logo-blue.jpg}
\cover{cover.jpeg}
\usepackage{multirow}
\usepackage{xpatch}
\makeatletter
\xpatchcmd{\chapter}
  {\if@openright\cleardoublepage\else\clearpage\fi}{\par\relax}
  {}{}
\makeatother

% 本文档命令
\usepackage{array, float}
\newcommand{\ccr}[1]{\makecell{{\color{#1}\rule{1cm}{1cm}}}}

% 修改标题页的橙色带
% \definecolor{customcolor}{RGB}{32,178,170}
% \colorlet{coverlinecolor}{customcolor}

\begin{document}

\maketitle
\frontmatter

\tableofcontents

\mainmatter

\chapter{用户手册}

\section{编译与测试说明}

请在项目根目录下执行以下命令完成编译:
\begin{lstlisting}
  make
\end{lstlisting}

\subsection{测试用例一、二}

对于第一个测试用例,请使用以下命令运行测试:
\begin{lstlisting}
  time ./test1 M
\end{lstlisting}

其中\verb|M|表示将区域划分为$M\times M$的网格。程序将会输出误差、求解时间,并将求解结果输出到\verb|result.txt|。在matlab中编写如下脚本即可绘制图像。
\begin{lstlisting}
  [x,y]=meshgrid(0:1/M:(1-1/M),0:1/M:(1-1/M));
  z = load("result.txt")';
  pcolor(x,y,z)
  shading interp;
\end{lstlisting}

对于第二个测试用例,运行与绘图方法与第一个测试用例完全相同,将\verb|test1|改为\verb|test2|即可。

\subsection{测试用例三}

\subsection{对流扩散方程求解器接口}

\subsection{INSE求解器接口}

\newpage

\chapter{数值积分}

\section{辛普森法则}

\section{6阶Newton-Cote法则}

\section{自适应积分}

\chapter{代数多重网格}

为节省工作量,本文所有求解器的解方程部分均采用代数多重网格。经测试,选取恰当的强依赖阈值,可以使代数多重网格的性能与几何多重网格相当,甚至超越几何多重网格\footnote{这项比较基于我的结果(代数多重网格)、樊睿的结果(几何多重网格)、凌子恒的结果(几何多重网格)。}。本章的内容来源于我在微分方程数值解课程中的多重网格大作业。

\section{算法描述}

\subsection{背景与介绍}

代数多重网格(AMG)是几何多重网格的一个自然推广,当我们发现由偏微分方程引出的离散方程组,使用多重网格具有如此优秀的表现时,自然会想,能不能将其推广到更一般的方程组,解更一般的大型稀疏矩阵。

答案是肯定的,这就是代数多重网格。具体而言,代数多重网格对$M$矩阵具有较好的表现,$M$矩阵是满足正定对称、对角元为正数、非对角元非正的矩阵。事实上,大部分椭圆方程的离散矩阵都是$M$矩阵。

对一个$n\times n$的矩阵$A^h$,我们不妨将$1,...,n$看作点,将$A^h$看作一个邻接矩阵,这样我们就得到了一个稀疏图,于是就有了网格结构。

要想应用多重网格的思路,我们必须将方程限制到粗网格中,求解后返回细网格调整。但此时我们面临着严峻的问题:我们不知道网格的结构,只知道一个矩阵。我们需要一些算法来选取粗网格点$\Omega^{2h}$,并且定义限制算子$I_{h}^{2h}$与插值算子$I_{2h}^{h}$,还要给出粗网格上的矩阵$A^{2h}$。

\subsection{选取粗网格点}

要想从中选取粗网格点,首先给出如下定义:

\begin{definition}
  若$i,j$满足:
  \begin{equation}
    |a_{i,j}|\geq \theta\max_{k\neq i}|a_{i,k}|
  \end{equation}

  则称$i$强依赖于$j$,也称$j$强影响$i$. 其中$\theta$是一个事先给定的阈值,称为“强依赖阈值”.

  记$N_i$为所有使$a_{i,j}\neq 0$的点$j$构成的集合,称为“相邻点”集。记$S_i$为所有强影响$i$的点构成的集合,$S_i^T$为所有强依赖于$i$的点构成的集合。
\end{definition}

在下文中,我们记粗网格$\Omega^{2h}=C$,细网格$\Omega^h=C\cup F$,即$F$表示所有在细网格但不在粗网格中的点。我们还记$C_i=S_i\cap C,\;D_i^s=S_i\cap F,\;D_i^w=N_i\setminus S_i$.

我们希望用一种\textbf{启发式的方法}来选取粗网格点,具体地,我们希望:
\begin{enumerate}[(1)]
  \item $\forall i\in F$,对于$j\in S_i$,或$j\in C_i$,或$j$强依赖于$C_i$中某点;
  \item $C$中的点尽可能多,但需要保持:$\forall i\in C$,没有$j\in C$使得$i$强依赖于$j$。
\end{enumerate}

当然,我们只是希望上述性质尽可能得到满足,为此,William L. Briggs的书上给出了\textbf{“染色算法”},如下:
\begin{enumerate}[1]
  \item $C,F\gets \varnothing$, $\lambda_i\gets |S_i^T|$.
  \item 选取$i\in\Omega^h\setminus(C\cup F)$,使$\lambda_i$最大.
  \item $C\gets C\cup\{i\}$, $F\gets F\cup (S_i^T\setminus C)$.
  \item $\forall j\in S_i^T,\;\forall k\in S_j\setminus(C\cup F)$, 令$\lambda_k\gets \lambda_k+1$.
  \item 若$C\cup F=\Omega^h$,结束,否则返回第2步.
\end{enumerate}

注意到上述过程需要维护一个$\lambda_i$数组,支持增加、删除、求最大下标三种操作,我们期望单次操作的复杂度不超过$O(\log n)$,因此我们实现了一个\textbf{二叉搜索树}。当然,使用线段树、平衡树也是可以的,不过因为可以提前建树,所以平衡操作是不必要的。

\subsection{构建插值算子}

对于$\mathbf{e}\in\Omega^{2h}$,插值算子具有如下格式:
\begin{equation}
  (I_{2h}^h \mathbf{e})_i=\left\{
    \begin{array}{ll}
      e_i & \text{若 }i\in C,\\
      \sum_{j\in C_i}\omega_{ij}e_j & \text{若 }i\in F.
    \end{array}
  \right.
\end{equation}

在粗网格中,我们求解的方程是$A^{2h}\mathbf{e}=\mathbf{f}$,其中$\mathbf{f}$是光滑化(即若干次G-S迭代)后的残差,通常比较小,因此
\begin{equation*}
  a_{ii}e_i\approx -\sum_{j\in N_i} a_{ij}e_j.
\end{equation*}

将$N_i$展开成三类,得
\begin{equation}
  a_{ii}e_i\approx -\sum_{j\in C_i} a_{ij}e_j -\sum_{j\in D^s_i} a_{ij}e_j -\sum_{j\in D^w_i} a_{ij}e_j.
\end{equation}

对于$j\in D^w_i$,将$e_j$近似为$e_i$,得
\begin{equation}
  \left(a_{ii}+\sum_{j\in D^w_i} a_{ij}\right)e_i\approx -\sum_{j\in C_i} a_{ij}e_j -\sum_{j\in D^s_i} a_{ij}e_j.
\end{equation}

对于$e_j\in D_i^s$,将其近似为
\begin{equation}
  e_j\approx \frac{\sum_{k\in C_i} a_{jk}e_k}{\sum_{k\in C_i} a_{jk}}.
\end{equation}

将(6.5)代入(6.4),得
\begin{equation}
  \omega_{ij}=-\frac{a_{ij}+\sum\limits_{m\in D_i^s}\left(\frac{a_{im}a_{mj}}{\sum\limits_{k\in C_i} a_{mk}}\right)}{a_{ii}+\sum\limits_{n\in D_i^w} a_{in}}
\end{equation}

本小节内容均来源于William L. Briggs的书,我们实现了上述过程,同时对反复使用的求和进行了预处理优化,能在$O(N_i)$的时间里完成单个$\omega_{ij}$的计算。

\subsection{构建限制算子与粗网格矩阵$A^{2h}$}

事实上,有了插值算子,我们可以直接由对称性和Garlerkin条件构建$I_{h}^{2h}$与$A^{2h}$,如下:
\begin{equation}
  I_{h}^{2h}:=(I_{2h}^{h})^T
\end{equation}
\begin{equation}
  A^{2h}:=I_{h}^{2h}A^hI_{2h}^{h}
\end{equation}

\subsection{V-Cycle}

实际上,代数多重网格的V-Cycle与几何多重网格基本一致,只不过使用的$A^h,I_h^{2h},I_{2h}^h$都是事先构建好的。为方便表述,记\verb|Rh|为$I_{h}^{2h}$,\verb|Ph|为$I_{2h}^{h}$,下面以递归形式描述。

\begin{lstlisting}
    VC(h, x, f)
        if size of grid <= 16
            return direct_solve(Ah, f)
        pre-smoothing for v1 times
        e = VC(2h, zeros, Rh*(f-Ah*x))
        x = x + Ph * e
        post-smoothing for v2 times
        return x
\end{lstlisting}

有了V-Cycle,自然可以定义FMG-Cycle.

\chapter{对流扩散方程的四阶MOL方法}

\section{设计要点}

\section{测试用例一}

\section{测试用例二}

\chapter{INSE的四阶近似投影方法}

\section{设计要点}

\section{测试用例三}

\chapter{总结}

\end{document}
