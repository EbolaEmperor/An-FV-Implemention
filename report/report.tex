\documentclass[lang=cn,10pt,bibend=bibtex]{elegantbook}

\title{Project: 四阶有限体积方法}

\author{Wenchong Huang}
\date{Aug 20, 2023}

\setcounter{tocdepth}{3}

\logo{logo-blue.jpg}
\cover{cover.jpg}
\usepackage{multirow}
\usepackage{xpatch}
\makeatletter
\xpatchcmd{\chapter}
  {\if@openright\cleardoublepage\else\clearpage\fi}{\par\relax}
  {}{}
\makeatother

% 本文档命令
\usepackage{array, float}
\newcommand{\ccr}[1]{\makecell{{\color{#1}\rule{1cm}{1cm}}}}

% 修改标题页的橙色带
% \definecolor{customcolor}{RGB}{32,178,170}
% \colorlet{coverlinecolor}{customcolor}

\begin{document}

\maketitle
\frontmatter

\tableofcontents

\mainmatter

\chapter{用户手册}

\section{编译与测试说明}

请在项目根目录下执行以下命令完成编译:
\begin{lstlisting}
  make
\end{lstlisting}

\subsection{测试用例一、二}

对于第一个测试用例,请使用以下命令运行测试:
\begin{lstlisting}
  time ./test1 M
\end{lstlisting}

其中\verb|M|表示将区域划分为$M\times M$的网格。程序将会输出误差、求解时间,并将求解结果输出到\verb|result.txt|。在matlab中编写如下脚本即可绘制图像。
\begin{lstlisting}
  [x,y]=meshgrid(0:1/M:(1-1/M),0:1/M:(1-1/M));
  z = load("result.txt")';
  pcolor(x,y,z)
  shading interp;
\end{lstlisting}

对于第二个测试用例,运行与绘图方法与第一个测试用例完全相同,将\verb|test1|改为\verb|test2|即可。但是程序将不会输出误差,需要用matlab读取解并用Richardson外插法计算误差。

\subsection{测试用例三}

对于第三个测试用例,请使用以下命令运行测试:
\begin{lstlisting}
  time ./test3 M Re Cr eps
\end{lstlisting}

其中\verb|M|表示将区域划分为$M\times M$的网格;\verb|Re|(正整数)表示雷诺数;\verb|Cr|(正实数)表示柯朗数;\verb|eps|(正实数)表示多重网格的迭代精度。注意,\verb|eps|并不是越小越好,过小会导致求解速度很慢但解的质量几乎没有提升。

程序将会输出误差、求解时间,并将求解结果输出到\verb|result.txt|。用matlab读取时,请使用以下脚本:
\begin{lstlisting}
  sol = load("result.txt");
  ux = reshape(sol(1,:),M,M);
  uy = reshape(sol(2,:),M,M);
  p = reshape(sol(3,:),M,M);
\end{lstlisting}

\subsection{对流扩散方程求解器接口}

以测试用例一为例,求解器的调用如下。

\begin{lstlisting}
  // 新建求解器
  FV_MOL_Solver solver;
  // 设置网格大小
  solver.setGridSize(stoi(argv[1]));
  // 设置终止时间
  solver.setEndTime(1.0);
  // 设置扩散系数
  solver.setNu(nu);
  // 设置时间步长,三个参数依次为:柯朗数、ux最大值、uy最大值
  solver.setTimeStepWithCaurant(1.0, 1.0, 0.5);
  // 设置外力项
  solver.setForcingTerm(&f);
  // 设置初值条件
  solver.setInitial(&phi);
  // 设置边值条件
  solver.setBondary("down", &phi, "Dirichlet");
  solver.setBondary("left", &phi, "Dirichlet");
  solver.setBondary("up", &dyphi, "Neumann");
  solver.setBondary("right", &dxphi, "Neumann");
  // 设置速度,当速度场为常向量时,用setConstVelocity能显著提速
  solver.setConstVelocity(1.0, 0.5);
  // 求解
  solver.solve();
  // 将解输出到文件
  solver.output("result.txt");
  // 计算误差。需要提供真解、范数,在norm.h中提供了p范数、无穷范数可供调用
  cout << "Error in max-norm: " << solver.checkerr(&phi, Norm_inf()) << endl;
  cout << "Error in 1-norm: " << solver.checkerr(&phi, Norm_p(1)) << endl;
  cout << "Error in 2-norm: " << solver.checkerr(&phi, Norm_p(2)) << endl;
\end{lstlisting}

以测试用例二为例,求解器的调用如下。
\begin{lstlisting}
  FV_MOL_Solver solver;
  solver.setGridSize(stoi(argv[1]));
  solver.setEndTime(10.0);
  solver.setNu(nu);
  solver.setTimeStepWithCaurant(1.0, 0.1, 0.1);
  solver.setInitial(&initphi);
  // 设置外力项为0
  solver.setNoForcingTerm();
  // 设置边值条件为周期
  solver.setPeriodicBondary();
  // 设置速度场(非常值)
  solver.setVelocity(&ux, &uy);
  solver.solve();
  solver.output("result.txt");
\end{lstlisting}

初值条件、边值条件、外力项使用的函数均为\verb|TimeFunction2D|的派生类,其原型的一部分如下
\begin{lstlisting}
class TimeFunction2D{
public:
    virtual double at (const double &x, const double &y, const double &t) const = 0;
    virtual double intFixX(const double &x, const double &d, const double &u, const double &t) const;
    virtual double intFixY(const double &y, const double &d, const double &u, const double &t) const;
    virtual double int2D(const double &l, const double &r, const double &d, const double &u, const double &t) const;
    virtual double accInt2D(const double &l, const double &r, const double &d, const double &u, const double &t) const;
    virtual double int2D_order6(const double &l, const double &r, const double &d, const double &u, const double &t) const;
    virtual double accInt2D_order6(const double &l, const double &r, const double &d, const double &u, const double &t) const;
};
\end{lstlisting}

用户的自定义函数必须继承\verb|TimeFunction2D|,并实现函数\verb|at(x,y,t)|,其返回值为用户自定义函数在$(x,y,t)$处的点值。若用户知道函数积分的解析表达式,也可以在子类中覆盖\verb|intFixX|(固定$x$对$y$积分)和\verb|intFixY|(固定$y$对$x$积分)。此外,若用户知道二重积分的解析表达式,建议用户将\verb|int2D|、\verb|accInt2D|、\verb|int2D_order6|、\verb|accInt2D_order6|全部覆盖。例如,测试用例二的$u_x$应该定义为:
\begin{lstlisting}
  class FUNCUX : public TimeFunction2D{
    public:
        double at (const double &x, const double &y, const double &t) const{
            return 0.1 * sin(pi*x) * sin(pi*x) * sin(2*pi*y);
        }
        double intFixX(const double &x, const double &d, const double &u, const double &t) const{
            return 0.1 * sin(pi*x) * sin(pi*x) * (cos(2*pi*d) - cos(2*pi*u)) / (2*pi);
        }
        double intFixY(const double &y, const double &d, const double &u, const double &t) const{
            return 0.1 * sin(2*pi*y) * (2*pi*(u-d) + sin(2*pi*d) - sin(2*pi*u)) / (4*pi);
        }
    } ux;
\end{lstlisting}

\subsection{INSE求解器接口}

INSE求解器的调用与对流扩散方程求解器类似。以测试用例三为例,INSE求解器的调用如下,我们在不同之处添加了注释。
\begin{lstlisting}
  INSE_Solver solver;
  solver.setGridSize(stoi(argv[1]));
  solver.setEndTime(0.5);
  solver.setNu(nu);
  solver.setTimeStepWithCaurant(stod(argv[3]), 3.0, 3.0);
  solver.setNoForcingTerm();
  // 设置初值(u为包含两个元素的TimeFunction2D指针数组)
  solver.setInitial(u);
  // 设置多重网格的迭代精度,若不设,则默认为1e-9
  solver.setEps(stod(argv[4]));
  solver.solve();
  solver.output("result.txt");
  // 计算误差,需要提供速度场真解(包含两个元素的TimeFunction2D指针数组)、压强真解(TimeFunction2D指针)、范数。checkerr的返回值为一个数组,包含3个元素,分别为ux、uy、p的数值解误差
  auto err = solver.checkerr(u, p, Norm_inf());
  cout << "Error in max-norm:\t" << err[0] << "\t" << err[1] << "\t" << err[2] << endl;
\end{lstlisting}

注意,我们的求解器只支持周期边界条件,因此不提供设置边界条件的接口。

\newpage

\chapter{数值积分}

\section{辛普森法则}

本文中的辛普森法则均指辛普森1/3法则,它是Newton-Cotes公式在$n=2$时的情形。辛普森法则用于估计如下形式的一维闭区间积分:
\begin{equation*}
  \int_a^b f(x) \;\text{d}x.
\end{equation*}

其估计如下:
\begin{equation*}
  I^S(f;a,b) = \frac{b-a}{6}\left(f(a)+4f\left(\frac{a+b}{2}\right)+f(b)\right).
\end{equation*}

对于辛普森法则的误差估计,我们有:
\begin{equation*}
  \left|I^S(f;a,b)-\int_a^b f(x) \;\text{d}x\right|\leq \frac{(b-a)^5}{2880}M,
\end{equation*}

其中$M$为$|f^{(4)}(x)|$的最大值。

\section{Boole法则}

Boole法则是Newton-Cotes公式在$n=4$时的情形。Boole法则对一维闭区间积分估计如下:
\begin{equation*}
  I^B(f;a,b) = \frac{b-a}{90}\left(7f(a)+32f\left(\frac{3a+b}{4}\right)+12f\left(\frac{a+b}{2}\right)+32f\left(\frac{a+3b}{4}\right)+7f(b)\right).
\end{equation*}

对于Boole法则的误差估计,我们有:
\begin{equation*}
  \left|I^B(f;a,b)-\int_a^b f(x) \;\text{d}x\right|\leq \frac{(b-a)^7}{945\times 2^{11}}M,
\end{equation*}

其中$M$为$|f^{(6)}(x)|$的最大值。

\section{二重积分}

对于二重积分
\begin{equation*}
  \int_a^b\int_c^d f(x,y) \;\text{d}y\;\text{d}x,
\end{equation*}

可以应用一维闭区间积分公式两次,得到对应的二重积分公式。例如,应用辛普森法则,我们有:
\begin{align*}
  I^{S,2D}(f;a,b,c,d)&= \left[\left(f(a,c)+4f\left(\frac{a+b}{2},c\right)+f(b,c)\right)\right.\\
  &+ 4\left(f\left(a,\frac{c+d}{2}\right)+4f\left(\frac{a+b}{2},\frac{c+d}{2}\right)+f\left(b,\frac{c+d}{2}\right)\right)\\
  &+ \left.\left(f(a,d)+4f\left(\frac{a+b}{2},d\right)+f(b,d)\right)\right] \times \frac{(b-a)(d-c)}{36}
\end{align*}

而误差估计可以为
\begin{equation*}
  \left|I^{S,2D}(f;a,b,c,d)-\int_a^b\int_c^d f(x,y) \;\text{d}y\;\text{d}x\right| = O((b-a)^5),
\end{equation*}

或
\begin{equation*}
  \left|I^{S,2D}(f;a,b,c,d)-\int_a^b\int_c^d f(x,y) \;\text{d}y\;\text{d}x\right| = O((d-c)^5),
\end{equation*}

对于Boole法则,有类似的结果。

\section{自适应积分}

我们希望把积分计算得更加精确,最好将误差控制在预先给定的$\varepsilon$以内。例如,在二维区域中,用某种法则计算积分值,记为$A$;然后将区域四等分,用同样的法则分别计算4个等分区域的积分值,记作$A_1,...,A_4$,然后计算误差:
\begin{equation*}
  E=|A-\sum_{i=1}^4 A_i|.
\end{equation*}

若$E<\varepsilon$,则将$\sum_{i=1}^4 A_i$作为结果返回,否则递归计算四个子区域。

\section{应用场景}

计算初值、真解的积分值时,我们将使用用自适应积分方法,将误差控制在$10^{-14}$以内。

考虑到在求解过程中计算面积分、体积分时应用自适应方法代价过大,因此求解过程中,遇到要算积分,我们将直接使用积分公式。对于一维平均积分,使用辛普森法则,有误差估计:
\begin{equation*}
  \frac{1}{h}\int_i^{i+h} g(x) \;\text{d}x=\frac{1}{h}I^S(g;i,i+h)+O(h^4).
\end{equation*}

对于二维平均积分,辛普森公式的阶数不够,因此采用Boole法则导出的二重积分公式,有误差估计:
\begin{equation*}
  \frac{1}{h^2}\int_i^{i+h}\int_j^{j+h} f(x,y) \;\text{d}y\;\text{d}x=\frac{1}{h^2}I^{B,2D}(f;i,i+h,j,j+h)+O(h^5).
\end{equation*}

\vspace{6em}

\chapter{代数多重网格}

为节省工作量,本文所有求解器的解方程部分均采用代数多重网格。经测试,选取恰当的强依赖阈值,可以使代数多重网格的性能与几何多重网格相当,甚至超越几何多重网格\footnote{这项比较基于我的结果(代数多重网格)、樊睿的结果(几何多重网格)、凌子恒的结果(几何多重网格)。}。本章的内容来源于我在微分方程数值解课程中的多重网格大作业,并额外添加了3.6节。

\section{背景与介绍}

代数多重网格(AMG)是几何多重网格的一个自然推广,当我们发现由偏微分方程引出的离散方程组,使用多重网格具有如此优秀的表现时,自然会想,能不能将其推广到更一般的方程组,解更一般的大型稀疏矩阵。

答案是肯定的,这就是代数多重网格。具体而言,代数多重网格对$M$矩阵具有较好的表现,$M$矩阵是满足正定对称、对角元为正数、非对角元非正的矩阵。事实上,大部分椭圆方程的离散矩阵都是$M$矩阵。

对一个$n\times n$的矩阵$A^h$,我们不妨将$1,...,n$看作点,将$A^h$看作一个邻接矩阵,这样我们就得到了一个稀疏图,于是就有了网格结构。

要想应用多重网格的思路,我们必须将方程限制到粗网格中,求解后返回细网格调整。但此时我们面临着严峻的问题:我们不知道网格的结构,只知道一个矩阵。我们需要一些算法来选取粗网格点$\Omega^{2h}$,并且定义限制算子$I_{h}^{2h}$与插值算子$I_{2h}^{h}$,还要给出粗网格上的矩阵$A^{2h}$。

\section{选取粗网格点}

要想从中选取粗网格点,首先给出如下定义:

\begin{definition}
  若$i,j$满足:
  \begin{equation}
    |a_{i,j}|\geq \theta\max_{k\neq i}|a_{i,k}|
  \end{equation}

  则称$i$强依赖于$j$,也称$j$强影响$i$. 其中$\theta$是一个事先给定的阈值,称为“强依赖阈值”.

  记$N_i$为所有使$a_{i,j}\neq 0$的点$j$构成的集合,称为“相邻点”集。记$S_i$为所有强影响$i$的点构成的集合,$S_i^T$为所有强依赖于$i$的点构成的集合。
\end{definition}

在下文中,我们记粗网格$\Omega^{2h}=C$,细网格$\Omega^h=C\cup F$,即$F$表示所有在细网格但不在粗网格中的点。我们还记$C_i=S_i\cap C,\;D_i^s=S_i\cap F,\;D_i^w=N_i\setminus S_i$.

我们希望用一种\textbf{启发式的方法}来选取粗网格点,具体地,我们希望:
\begin{enumerate}[(1)]
  \item $\forall i\in F$,对于$j\in S_i$,或$j\in C_i$,或$j$强依赖于$C_i$中某点;
  \item $C$中的点尽可能多,但需要保持:$\forall i\in C$,没有$j\in C$使得$i$强依赖于$j$。
\end{enumerate}

当然,我们只是希望上述性质尽可能得到满足,为此,William L. Briggs的书上给出了\textbf{“染色算法”},如下:
\begin{enumerate}[1]
  \item $C,F\gets \varnothing$, $\lambda_i\gets |S_i^T|$.
  \item 选取$i\in\Omega^h\setminus(C\cup F)$,使$\lambda_i$最大.
  \item $C\gets C\cup\{i\}$, $F\gets F\cup (S_i^T\setminus C)$.
  \item $\forall j\in S_i^T,\;\forall k\in S_j\setminus(C\cup F)$, 令$\lambda_k\gets \lambda_k+1$.
  \item 若$C\cup F=\Omega^h$,结束,否则返回第2步.
\end{enumerate}

注意到上述过程需要维护一个$\lambda_i$数组,支持增加、删除、求最大下标三种操作,我们期望单次操作的复杂度不超过$O(\log n)$,因此我们实现了一个\textbf{二叉搜索树}。当然,使用线段树、平衡树也是可以的,不过因为可以提前建树,所以平衡操作是不必要的。

\section{构建插值算子}

对于$\mathbf{e}\in\Omega^{2h}$,插值算子具有如下格式:
\begin{equation}
  (I_{2h}^h \mathbf{e})_i=\left\{
    \begin{array}{ll}
      e_i & \text{若 }i\in C,\\
      \sum_{j\in C_i}\omega_{ij}e_j & \text{若 }i\in F.
    \end{array}
  \right.
\end{equation}

在粗网格中,我们求解的方程是$A^{2h}\mathbf{e}=\mathbf{f}$,其中$\mathbf{f}$是光滑化(即若干次G-S迭代)后的残差,通常比较小,因此
\begin{equation*}
  a_{ii}e_i\approx -\sum_{j\in N_i} a_{ij}e_j.
\end{equation*}

将$N_i$展开成三类,得
\begin{equation}
  a_{ii}e_i\approx -\sum_{j\in C_i} a_{ij}e_j -\sum_{j\in D^s_i} a_{ij}e_j -\sum_{j\in D^w_i} a_{ij}e_j.
\end{equation}

对于$j\in D^w_i$,将$e_j$近似为$e_i$,得
\begin{equation}
  \left(a_{ii}+\sum_{j\in D^w_i} a_{ij}\right)e_i\approx -\sum_{j\in C_i} a_{ij}e_j -\sum_{j\in D^s_i} a_{ij}e_j.
\end{equation}

对于$e_j\in D_i^s$,将其近似为
\begin{equation}
  e_j\approx \frac{\sum_{k\in C_i} a_{jk}e_k}{\sum_{k\in C_i} a_{jk}}.
\end{equation}

将(6.5)代入(6.4),得
\begin{equation}
  \omega_{ij}=-\frac{a_{ij}+\sum\limits_{m\in D_i^s}\left(\frac{a_{im}a_{mj}}{\sum\limits_{k\in C_i} a_{mk}}\right)}{a_{ii}+\sum\limits_{n\in D_i^w} a_{in}}
\end{equation}

本小节内容均来源于William L. Briggs的书,我们实现了上述过程,同时对反复使用的求和进行了预处理优化,能在$O(N_i)$的时间里完成单个$\omega_{ij}$的计算。

\section{构建限制算子与粗网格矩阵$A^{2h}$}

事实上,有了插值算子,我们可以直接由对称性和Garlerkin条件构建$I_{h}^{2h}$与$A^{2h}$,如下:
\begin{equation}
  I_{h}^{2h}:=(I_{2h}^{h})^T
\end{equation}
\begin{equation}
  A^{2h}:=I_{h}^{2h}A^hI_{2h}^{h}
\end{equation}

\section{V-Cycle}

实际上,代数多重网格的V-Cycle与几何多重网格基本一致,只不过使用的$A^h,I_h^{2h},I_{2h}^h$都是事先构建好的。为方便表述,记\verb|Rh|为$I_{h}^{2h}$,\verb|Ph|为$I_{2h}^{h}$,下面以递归形式描述。

\begin{lstlisting}
    VC(h, x, f)
        if size of grid <= 16
            return direct_solve(Ah, f)
        pre-smoothing for v1 times
        e = VC(2h, zeros, Rh*(f-Ah*x))
        x = x + Ph * e
        post-smoothing for v2 times
        return x
\end{lstlisting}

有了V-Cycle,自然可以定义FMG-Cycle.

\section{纯Neumann条件或周期条件下的求解}

对于纯Neumann条件或周期条件下的二维Poisson方程,我们知道,在解存在的前提下,任何两个解都相差一个常数。此时直接使用多重网格求解也是可以的,但是解会发生“漂移”,即平均值越来越大。这对数值方法很不友好,实际测试时,一个$512\times 512$的二维网格所需的FMG迭代次数可以高达$30$次。为此,我们人为添加“平均值为$0$”的额外条件,并定义下面这个操作为\textbf{“均值标准化”}。

\begin{equation*}
  \mathbf{v}\gets \mathbf{v}-\left(\frac{1}{n}\sum_{i=1}^n v_i\right)\mathbf{e}.
\end{equation*}

我们在第一层网格中,每进行完一次smoothing都对当前解执行一次均值标准化,实测效果极佳,可以将FMG迭代次数降低到5次以内。这个做法可以在文献\cite{Multigrid}的第113页找到。

\textbf{在INSE的近似投影算子中,我们将会用到本节的做法}。

\newpage

\chapter{对流扩散方程的四阶MOL方法}

\section{设计要点}

\section{测试用例一}

\section{测试用例二}

\chapter{INSE的四阶近似投影方法}

\section{设计要点}

\section{测试用例三}

\chapter{总结}

\printbibliography[heading=bibintoc,title=\ebibname]

\end{document}
